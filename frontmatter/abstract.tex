\thispagestyle{plain}
% \section*{Abstract}

\begin{table}[!h]
\centering

\begin{tabu}{X[0.1,c] X[0.5,c] X[0.1,c]}
    \addcontentsline{toc}{chapter}{Abstract}
    &\textbf{Abstract}&\\[1em]
    &Pedestrian Dynamics: Ubiquitous Sensing, Interactions, and Models&\\[0.7em]
    &by&\\[0.7em]
    &Arash Kalatian&\\[0.7em]
    &PhD in Civil Engineering - Department of Civil Engineering&\\[0.7em]
    &Ryerson University&\\[0.7em]
    &2020&\\[1em]
    % &Dr Bilal Farooq, Supervisor&\\[1em]
\end{tabu}
\end{table}


% \begin{abstract}
The dissertation outlines novel analytical and experimental methods for discrete choice modelling using generative modelling and information theory.
It explores the influence of information heterogeneity on large scale datasets using generative modelling.
The behaviourally subjective psychometric indicators are replaced with a learning process in an artificial neural network architecture.
Part of the dissertation establishes new tools and techniques to model aspects of travel demand and behavioural analysis for the emerging transport and mobility markets.
Specifically, we consider: (i) What are the strengths, weaknesses and role of generative learning algorithms for behaviour analysis in travel demand modelling?
(ii) How to monitor and analyze the identifiability and validity of the generative model using Bayesian inference methods?
(iii) How to ensure that the methodology is behaviourally consistent?
(iv) What is the relationship between the generative learning process and realistic representation of decision making as well as its usefulness in choice modelling? and
(v) What are the limitations and assumptions that have needed to develop the generative model systems?

This thesis is based on four articles introduced in \cref{chap2,chap3,chap4,chap6}. \cref{chap2,chap3} introduces a restricted Boltzmann machine learning algorithm for travel behaviour that includes an analysis of modelling discrete choice with and without psychometric indicators.
\cref{chap4} provides an analysis of information heterogeneity from the perspective of a generative model and how it can extract population taste variation using a Bayesian inference based learning process.
One of the most promising applications for generative modelling is for modelling the multiple discrete-continuous data.
In \cref{chap6}, a generative modelling framework is developed to show the process and methodology of capturing higher-order correlation in the data and deriving a process of sampling that can account for the interdependencies between multiple outputs and inputs.
A brief background on machine learning principles for discrete choice modelling and newly developed mathematical models and equations related to generative modelling for travel behaviour analysis are provided in the appendices.
% \end{abstract}
\clearpage
