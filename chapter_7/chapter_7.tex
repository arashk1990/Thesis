\part{ Concluding Remarks}

\chapter{Conclusion}
\label{chap7}
This dissertation provides a series of interconnected research studies that address pedestrian issues through a lens of modern data sources and the new data-driven paradigm of machine learning. By introducing alternative ways of data collection, data-driven approaches of analyzing the collected data, interpretability methods for explaining the models, and applying the frameworks to a series of futuristic scenarios, this work tries to provide solutions to prepare the urban space for disruptive emerging technologies. \cref{chap1b} presents the background of pedestrian research in two fronts: alternative data sources to traditional surveys, and the studies that concern the interactions of pedestrians and automated vehicles. The methodological history of pedestrian crossing behaviour, namely wait time and trajectory, is provided in this chapter, as well as the new trends concerning the use of new data-driven machine learning models to analyze pedestrian behaviour. The chapter highlights the strengths and weaknesses of current and proposed methods of pedestrian crossing behaviour modelling.

The first study concerns the use of Wi-Fi data, as a novel data source enabling large-scale passive collection in congested urban areas. By developing a deep residual neural network in a semi-supervised manner in \cref{chap2}, mode of transportation of the network users can be detected in highly congested urban areas, enabling tracking and counting pedestrians. The semi-supervised nature of the algorithm helps to improve the performance of the model by using unlabelled, easily-collected, low-cost Wi-Fi data.

The second work concerns the use of virtual reality, another novel source of data as an alternative to stated-preference surveys. In \cref{chap3}, VIRE, an interactive virtual reality framework is introduced and utilized to analyze the distracting effect of smartphones on crossing pedestrians.

The third work focuses on the interactions of pedestrians and automated vehicles, and the policy suggestions and practice recommendations towards the transition to fully automated environments. \cref{chap4} introduces a large-scale data collection campaign using VIRE, during which the crossing behaviours of pedestrians are observed and assessed in the presence of automated vehicles. By developing neural network-based survival models, this chapter seeks to propose a machine-learning based version of a well-known traditional model. Interpretability of the proposed models is also included in this chapter, to compensate for the \textit{black box} nature of neural networks as the major barrier for their extensive use in some areas.
Furthermore, \cref{chap6} tackles the same problem of pedestrian crossing behaviour in automated environments with a focus on trajectory. This chapter highlights how open-source video datasets of available test AVs are still lacing the essential information for pedestrian studies. By selecting variables from VIRE that can be captured by a hypothetical automated vehicle, this chapter provides insight on the required data for pedestrian trajectory prediction. 

\section{Research Contributions}
\label{chap7:sec1}
This dissertation puts together four research articles, each making a different contribution to the state-of-the-art in pedestrian research.
The following summarizes the main results from the articles based chapters of \cref{part_2:dissertation_articles}:

\subsubsection*{A semi-supervised deep residual network for mode detection in Wi-Fi signals}
The third chapter of this dissertation opens with the idea of using labelled and unlabelled Wi-Fi data, as an alternative to traditional travel surveys, video tracking, and recently popular GPS data.
We developed a novel framework by incorporating a very deep residual network, trained using pseudo-labelling, which enables using the collected unlabelled Wi-Fi data and captures the high levels of nonlinearity within the data.
This approach is is useful as it collects and utilizes data from network users, even if they are not directly part of the experiment, loosing the constraints of GPS data of necessity of involvement of participants by installing an app.
The proposed framework is an application of artificial intelligence techniques that has the potential to benefit from ubiquitous communication networks to improve smart transportation planning, operations, and design. One advantage of the proposed framework is its possibility to be applied to data from other location-aware technologies, with minimum modifications required. 
Infrastructures for collecting Bluetooth and Wi-Fi signals are already implemented in Toronto and several other cities in the world. Thus, data used in such studies can be retrieved from the municipalities of cities, without the need of any extra facility and infringing privacy policy of users. 


\subsubsection*{Analysis of distracted pedestrians' waiting time: Head-Mounted Immersive Virtual Reality application}
The fourth chapter begins with the introduction of VIRE, a virtual reality immersive environment which enables collecting data in safe and controlled environments. VIRE is proposed as a more realistic alternative to state preference surveys, which enables collecting data for scenarios that are not easy to implement in real world settings. As an initial application of VIRE, and as it may be unsafe to track pedestrians' behaviours while they are distracted using their phones, VIRE appears to be an alternative with minimal risk and great potential. The application of traditional survival analysis models is used in this chapter prior to the shift towards data-driven survival models in the next sections. 


\subsubsection*{Decoding pedestrian and automated vehicle interactions using immersive virtual reality and interpretable deep learning}
The fifth chapter begins the deep investigation of interactions of pedestrians and automated vehicles. By focusing on the behaviours of pedestrians crossing at unsignalized mid-block intersections, wait time before initiating a cross is analyzed in this chapter. An enhanced version of survival models is used to understand the effects of various covariates on the wait time, followed by a post-hoc game theory interpretability model.
This chapter contributes to the methodology of survival analysis by developing a data-driven version of Cox Proportional Hazards model, empowered with neural networks to capture higher levels of nonlinearities among the data. Another contribution of this chapter to transportation research community is recommendation of policy suggestions and practical implications on two levels: virtual reality data collection and pedestrian crossing behaviour. 


\subsubsection*{Behavioural modelling of pedestrian movement using novel data sources and machine learning}
In the sixth chapter, data-driven time-series data is developed to utilize the data from initial movements of pedestrians as well as contextual information of the environment to predict the trajectory of pedestrians. 

This methodological framework allows for using information of the crossing environment as auxiliary information, and enhance the performance of the models based on them. By adding auxiliary data, the proposed framework will be able to take into account the effects of road specifications, i.e. lane width and type of road, traffic parameters, i.e. speed limit, arrival rate, and environmental conditions, i.e. weather conditions and time of the day. By choosing the contextual information in a way that can be obtained by AVs, suggestions are provided for future data releases of automated vehicles to account for pedestrian-oriented points of view.

\section{Limitations}
\label{chap7:sec2}
Every analysis approach has its own set of advantages and drawbacks, and it is the analyst's job to be aware of them and choose the right model and data in the right situation. Regarding the use of Wi-Fi data, the proposed framework of distinguishing pedestrians can be helpful in counting and tracking pedestrians origin and destination in large scales, however lacks the ability to track the exact trajectory followed by the user. The accuracy of the Wi-Fi sensors limits spotting pedestrian locations within a sphere of approximately 50 meters. The proposed model, although applicable to other sets of data with some modifications, still can be updated to more data-agnostic models in order to be transferable to other data sources. In addition, data collection was limited to certain dates, and the effect of weather conditions on the accuracy of the model was not investigated. Another common issue with studies on pedestrian detection is their high computational power required, which makes them difficult to apply to real-world applications in real time. The semi-supervised model developed in this dissertation, although trained locally, is not designed for real time counting of pedestrians or other modes of transportation.

Regarding the virtual reality data used for the studies of this dissertation, several technological limitations exist as VR is a rapidly developing technology on its first days. Considering pedestrians in group, which have proven to be an important factor on determining pedestrian behaviour, is still difficult to implement in virtual reality environment. On the graphics and reality of the scenarios, significant improvements can be done to make the scenarios as naturalistic as possible. The data collection campaign was designed to be as diverse as possible, with various locations included in the process. However, VR technology is yet to be fully developed to be accessible to people with special needs.   

Model interpretation of machine learning frameworks is a relatively new topic of interest and all the proposed methods in the literature have serious drawbacks in different aspects. The interpretability methods used in this dissertations are as well prone to the fundamental problems of similar methods, and with the advances of these methods, a reconfiguration of the models can be conducted. 

While this research has tried to address pedestrian issues in the context of novel datasets and methodologies, it has only been able to apply the methods in specific problems of pedestrians. The datasets, and the methodologies, have yet to be expanded substantially to include various other data sources and issues of pedestrian research.

\section{Future Work}
\label{chap7:sec3}
The research presented in this dissertation can be extended and built upon in different directions. For the passive collection of Wi-Fi data, incorporating other data sources, labelled and unlabelled, and providing data-agnostic hybrid models to detect and tract network users can be a direction to follow and achieve higher accuracies. By developing a mobile application and ask user permissions for their data, the quality of the data can be increase, with incentives such as promoting greener modes of transportation, counting carbon footprints and providing points for monetary awards in exchange for more steps can be some of the future directions. Moreover, aside from detection and counting, behaviour of pedestrians can be modelled to analyze the contributing factors in mode choice, pedestrian route choice, activities, etc., and to predict the next locations or activities of pedestrians.

The interactions of automated vehicles and pedestrians is another part of this dissertation with various possible extensions. The virtual reality experiments conducted are aimed at providing a framework for comprehensive investigation of the interactions between the two agents. Wait time and trajectory, under specific conditions, was parts covered in this dissertation. Training an automated vehicle based on the collected data can be another direction to follow, and develop pedestrian-friendly braking systems for AVs. Extending the VR experiments, by including group crossings, accessibility services for people with disabilities, and better qualities in the graphics of the scenarios, is another important area that needs to be explored. Analyzing communications of vehicles and pedestrians, and developing and testing new audio or visual communication can help the pedestrians crossing more conveniently. The effectiveness of such methods can also be tested in the virtual reality environment. 

Future research can further enhance the interpretability of the models, by including interactive effects of variables to the outcome, and developing other methods of machine learning interpretability. Machine learning models can only fully replace traditional models when they are able to provide the same level of interpretability. Despite the recent advances on the subject, interpretability methods are a thriving field and new methods should be tested continuously. By providing solid interpretability results from different areas of the interactions between pedestrians and AVs, a more clear understanding of future urban areas, and the dynamics of pedestrians in that context can be elaborated.

The final area for future work would be to adapt these machine learning algorithms and transfer the models developed in this dissertation to real-world video data available by AV manufacturers. As discussed in this dissertation, these datasets are yet relatively small, and without all the required information to focus on the pedestrians. A benchmark dataset, including the data related to pedestrians, seems to be of vital importance to help researchers train and test their models in a more consistent and comprehensive way. Virtual reality data collected in this study can form an important part of this benchmark dataset. 


