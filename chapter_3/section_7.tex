\section{Conclusions and Future Work}
\label{S:D7}
By developing an immersive virtual reality-based method for collecting data of pedestrian waiting time in street crossing paradigm, data can be collected in a safe and controlled environment. To observe the effects of various distraction parameters on pedestrian waiting time, cox proportional hazards model was adopted for the modelling purpose. 

With regards to the model, results show higher minimum missed gap, dangerous PET (i.e. less than 1.5 seconds), crossing duration, initial crossing speed, percentage of time head orientation towards the phone during wait time and crossing, and gender. 

Our study is not without limitations which can be addressed in future studies. Eye movements, brain activity and heart beats can be measured as indicators of participants’ stress and distraction level. Other types of distraction can also be added to the experiment. In terms of dataset, larger datasets can be collected and analysed to remove the possible biases. Other safety measures can also be explored and compared to each other to better analyse the effect of different safety measures. Finally, waiting time for different types of intersections can also be studied.
