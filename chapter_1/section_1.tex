\chapter{Overview}\label{ch:over}
The world of transportation is on the cusp of a great technological revolution. Unprecedented changes are happening in various fronts, each of which promising to address deep-rooted challenges of current transportation networks. These changes can be summarized into three categories:
\begin{itemize}
    \item \textbf{Big data revolution:} Availability of very big datasets in recent years, with the boosted power in computational power has made it possible to acquire information on what was impossible before. From the fine-grained details of the behaviours of individuals, to massive aggregated data of the whole population of a country, we can now know about transportation behaviours, preferences and choices of people more than any time in history.
    \item\textbf{\textbf{Machine learning revolution:}} Availability of data and computational power has not only served as a way to provide better description of the situation. Data-driven modeling approaches, and in particular Machine Learning (ML) models, had been introduced in the middle of the 20th century. But the lack of effectiveness of the computers resulted in the so-called \textit{winter of Artificial Intelligence (AI)} in the 70s~\cite{russell2002artificial}. However, only in the last decade, and with the improvements in computational power and datasets, advanced machine learning and deep learning models have become feasible to develop and train~\cite{Goodfellow-et-al-2016}.
    \item\textbf{Advent of Automated Vehicles:} Automated vehicles are being tested in various levels by various research centers and companies around the globe. Once merely futuristic concept of science fiction, the presence of automated vehicles on urban roads is now only a matter of time. The introduction of fully or partially automated vehicles to urban and rural networks leads to drastic changes in mobility~\cite{angerholzer2017autonomous}, congestion~\cite{djavadian2020multi}, greenhouse gas emissions~\cite{greenblatt2015autonomous}, travel behaviours~\cite{farooq2018virtual}, parking facilities~\cite{bahrami2019impacts}, etc. 
\end{itemize}
With the aforementioned changes on the way, new dynamics are expected to form in urban areas, among different network users. This dissertation is an attempt to explore the dynamics of pedestrians in the age of the technological revolutions in transpiration. To provide and enhance the quality of pedestrian related services, it is of vital importance to understand, regenerate and predict parameters that characterize pedestrians' traffics and behaviours \cite{nikolic2016probabilistic}. 

The first step to understand the characteristics of pedestrians would be to collect pedestrians' data. Whether using the collected information for simulation, or merely for estimating the number walking population in an area, the size and quality of the collected data can affect the outcome of the study greatly. Conventional methods of data collection for pedestrian studies typically consists of manual counting methods and travel surveys from limited number of participants. Intrinsic problems of traditional methods include involvement of human errors and biased responses, as well as being time-consuming, expensive and not representative~\cite{shen2014review,dollar2012pedestrian}. Advances in different fields of technology in recent decades have made it possible to automatically detect and track pedestrians at large scale~\cite{bauer2009measurement}. As a result, novel methodologies to collect data have gained popularity among scholars in different fields~\cite{farooq2015ubiquitous}. Among these new methods, automatic video tracking of pedestrians has been widely used, as it has the potential to detect data of each and every pedestrian crossing the video frame. However, myriad of challenges are yet to be considered in this method, such as occlusion, grouping, and  various visual effects (e.g. lighting, shadows, distortion, non-human moving objects)~\cite{ettehadieh2014automated}. 
Rapid growth of smartphones, on the other hand, has provided a unique opportunity for ubiquitous and pervasive data collection for transportation purposes. Penetration rate of smartphones among adults has reached over 70\% in developed countries~\cite{pen}, making it possible to benefit from multiple sensors they carry for observing their users' behaviours. Another novel method for data collection with significant improvements in recent years is the Virtual Reality (VR) technology. In conventional field experiments on pedestrian behaviours, it is often difficult to implement different scenarios as the results may be disastrous in terms of the participants’ safety. In addition, it may be expensive or unsafe to repeat an experiment with the exact same traffic conditions to capture the effects of an implemented safety measure. These problems, along with the fact that some scenarios are impossible to be recreated in technological terms, have led researchers to use pictures, videos and photomontages to assess participants perception. With the development of interactive computer-generated experiences, VR experiments have gained popularity in various research fields. Using a head mounted VR display device, virtual scenes can be directly projected to the participants so that the they will be immersed in the simulated environment~\cite{farooq2018virtual}. Scenarios used in experiments may be not safe or expensive to apply on real roads, due to reasons such as dangerous implementations or lack of infrastructures. VR simulator allows running such scenarios, along with scenarios containing new technologies or services that participants have limited mental image of them. 

The second step of the journey to understand pedestrians' dynamics would be to analyze the collected data and extract meaningful patterns from them. Availability of detailed, high-dimensional, complex datasets today, as well as the high computational powers, have led to the rapid and ubiquitous emergence of data-driven approaches. In many cases, traditional models cease to capture the high level of nonlinearities among the data. Thus, utilizing machine learning models, as well as developing ML-based versions of traditional models, have been an inevitable trend in modeling in recent years. Despite the strength of machine learning models in increasing the prediction power of the models, they often fail to provide the descriptive power required for decision making and policy planning. Interpretability and explainability of most of ML algorithms are often considered as a cumbersome or even unnecessary task, as making models more interpretable is often associated with a loss of prediction power. In the areas of research that gaining information on \textit{why?} and \textit{how?} the model works is as important as the predictivity of the model, traditional explainable models are mostly favored. Thus, in order to apply an ML model to solve a problem, one must first find out a way to interpret the results of the model. Various approaches to make machine learning models explainable have emerged in recent years. In addition to inherently explainable ML models, e.g. decision trees, various model-agnostic methods have been introduced to address the \textit{black-box} nature of machine learning algorithms, as post-hoc methods relying only on the input and the output of the models~\citep{molnar2019interpretable}.

Considering the new dynamics of urban areas with the emergence of automated vehicles, the behaviour of pedestrians as a vital part of the network needs to be carefully investigated for possible changes. Using two previous steps, features of future streets, policy suggestions and recommendations, and automated vehicles’ traffic parameters that relate to pedestrians can be explored before the transition towards fully automated environments.     

This study explores the idea of utilizing new technologies for pedestrian related studies in different aspects, i.e. from sensing the presence and counting the number of pedestrians, to developing models to predict and explain pedestrians' behaviours in automated environments. The main objective of this research is to overcome the challenges that typical data collection methods impose on pedestrian studies. Collecting high-dimensional, fine-grained data from the pedestrians eases the path to explore and develop advanced data-driven machine learning models. By utilizing modern datasets and modeling approaches, we then seek to understand, predict, and explain the behaviours of pedestrians in future urban areas.


\section{Research Contributions}
This dissertation tries to tackle one of the most important challenges in transportation studies, with a focus on pedestrian related research: future urban areas and the unpredictable challenges that will show up in transforming to smart cities. The work in this dissertation will contribute to the following aspects:


\begin{itemize}
    \item \textbf{Pedestrians' data collection}
    \begin{itemize}
         \item Introducing passively collected Wi-Fi data as a reliable alternative to traditional methods for pedestrian detection
         \item Utilizing unlabelled data to achieve high performance with lower costs
         \item Pedestrian mobility detection in a highly congested areas using the available infrastructure on the cities
         \item Introducing and implementing a Virtual Reality data as a novel source of data to obtain highly detailed data in customized hypothetical scenarios in a controlled and safe environment
    \end{itemize}
\end{itemize}

\begin{itemize}
    \item \textbf{Behavioural studies of pedestrians}
    \begin{itemize}
    \item Analyzing behaviours of pedestrians in automated environments of future urban areas
    \item Analyzing the effect of smartphone distraction on pedestrians crossing behaviour
    \item Exploring steps towards knowing the characteristics of future urban areas from a pedestrian-friendly perspective
    \item Introducing a data-driven based method to predict the trajectory of pedestrians in future urban areas
\end{itemize}
\end{itemize}

 \begin{itemize}
    \item \textbf{Data-driven methodologies}
    \begin{itemize}
      \item Developing a semi-supervised deep residual framework for mode classification
      \item Introducing a neural network based survival model to capture the nonlinearities of the high dimensional data
      \item Implementing post-hoc model interpretability frameworks to explain the contributing factors to the model outputs
    \end{itemize}

\end{itemize}  
    

    
   

