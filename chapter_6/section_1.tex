\section{Introduction}
\label{S:TInt}

Analyzing pedestrians' crossing behaviour is a topic of interest in transportation studies. Predicting pedestrian behaviour in different circumstances contributes to better planning, design, and operations of traffic control infrastructure as well as improved pedestrian safety and traffic flow. With the widespread adoption of autonomous vehicles in urban areas and their development in recent years, behaviour of pedestrians is expected to be affected dramatically. In particular, communications between pedestrians and drivers, i.e. eye contact, head and body movements that play an important, yet silent, role in pedestrian crossing choices and behaviours, will no longer be available in the same form in an autonomous environment. To be able to maintain similar types of communications between pedestrians and vehicles in an autonomous environment, they need to be able to capture pedestrians' reactions, and predict further intentions, choices, movements and trajectories based on them. Failures in predicting moving pedestrian's behaviour and taking necessary actions in time have resulted in catastrophic accidents of autonomous vehicles in recent years, even at very slow speeds~\citep{uber,vienna}.

In this study, we focus on the trajectory prediction of pedestrians while crossing a road in automated environment, given an initial observation of behaviour as well as contextual information of the environment that the movement takes place. Naturalistic data for this study is collected using the Virtual Immersive Reality Environment (VIRE) \citep{farooqvire} framework. Virtual Reality (VR) tools have made it possible to create an immersive and controlled environment. Scenarios used in experiments may be difficult or impossible to apply on real roads, due to reasons such as dangerous implementations or lack of infrastructures. VR simulator allows running such scenarios, along with scenarios containing new technologies or services that participants have limited mental image of, or expressing their features in textual context is difficult. Conducting a controlled experiment enables us to record participants' reactions and other related variables.

To model trajectories of pedestrians while crossing a road in automated environment, a novel multi-input network of Long Short-Term Memory (LSTM) and fully connected dense layers is developed in this study. In the proposed model, time-series data of the initial steps of crossing, including trajectories, head orientations, and distance to vehicles , are added to non-time-series data of contextual information of the environment in which crossing occurs, to predict multi-step outputs of pedestrian trajectories. To the best of our knowledge, this is the first such model in the behavioural choice modelling as well as the related applications of machine learning.

The rest of this paper is organized as follows: A review of previous studies on the subject is provided in the next section. Section \ref{S:T3} discusses the data collection and pre-processing procedures in detail. Methodology and proposed architecture are described in section \ref{S:T4}. Results and the application of our proposed framework on the data are provided in section \ref{S:T5}. Finally, section \ref{S:T6} discusses conclusions, final remarks and future research plans of the project.