\section{Literature Review}
\label{S:w2}
Data collection methods that are based on location-aware technologies, and thus the associated mode detection studies, can be divided into two main categories: user-centric and network-centric methods. In \emph{user-centric} methods, users are required to be actively involved in the data collection procedure. Examples include GPS and accelerometer data, or a combination of both that can be used to infer transportation mode \cite{zheng2008understanding,reddy2008determining,stenneth2011transportation,endo2016deep,xiao2017identifying,dabiri2018inferring,efthymiou2019transportation}. 
Zheng~\textit{et al.}~\cite{zheng2008understanding} used GPS data to detect mode of transportation. In their study, features such as heading change rate, stop rate and velocity change rate were defined. A graph-based post-processing algorithm was proposed considering the conditional probability between different modes of transport. An accuracy of 76\% was achieved in this study in predicting the transportation mode correctly. Some studies have tried using multiple data sources to better observe network users' behaviours. Reddy~\textit{et al.}~\cite{reddy2008determining} implemented GPS data along with smartphones accelerometer data to identify walking, running, biking and motorized modes. An accuracy of over 90\% was achieved using two-stage decision tree and discrete Hidden Markov Model. The data collection procedure of this study involved attaching 5 phones to each participant simultaneously, which seems to be impossible to implement for real-life large scale data collection. Moreover, training classifiers based on data from 5 out of 6 participants and validating them on data from the other participant, resulted in a drop in accuracy by 10\%, which may have been due to the overfitting to training data. Stenneth~\textit{et al.}~\cite{stenneth2011transportation} exploited information from underlying transportation network to distinguish users' mode of transport between stationary, walking, biking, driving, and using public transit. Classifiers such as decision tree, random forest, and Multilayer Perceptron were developed in this study. Training classifiers using solely GPS data resulted in an accuracy of 75\%. However, when network information such as bus routes, rails, real-time bus schedules, etc, were added, random forest method was shown to have an accuracy of 93.7\%. In a more recent study, Bantis and Haworth~\cite{bantis2017you} incorporated multiple smartphone data, as well as sociodemographic characteristic of travelers to develop a mode detection framework using Hidden Markov Models. Similarly, Yazdizadeh~\textit{et al.}~\cite{yazdizadeh2019automated} trained three random forest classifies on GPS data enhanced with network information and social media data. Despite the higher accuracy achieved when adding sociodemographic information or social media data, gathering such data is not always feasible and can only be used as a complementary tool, and not as a replacement, to mobility surveys. Moreover, the heterogeneity in the accuracy of sensors adds another level of challenge to the accurate position prediction when some sensors are not accessible, thus making it difficult for this approach to introduce a generalized solution. Xiao~\textit{et al.}~\cite{xiao2017identifying} outperformed traditional decision tree-based models by applying a tree-based ensemble classification algorithms using over 100 GPS trajectory augmented features .
Although user-centric approaches potentially tend to collect more precise and accurate data, the collection relies upon a specific and limited number of participants. This makes such methods hard to implement in large scales and to address real-life transportation problems. Such data collection method may result in biased data due to the involvement of only a certain type of participants. In addition, extra operational costs are usually associated with such studies, as they require mobile Apps, participant's time, and a high level of battery consumption. Recently, Efthymiou~\textit{et al.} tried to address the battery consumption problem by using accelerometer, gyroscope and orientation data sensors of smartphones, however,the involvement of participants still remains an issue.
On the contrary, \emph{network-centric} methods try to collect data passively, requiring no intervention from the users of the network. Main sources of data in network-centric approaches in the literature have been Wi-Fi data, Bluetooth transceivers data, and GSM signals data~\cite{sohn2006mobility,wang2010transportation,mun2008parsimonious,wang2010transportation}. Sohn~\textit{et al.}~\cite{sohn2006mobility} used coarse-grained GSM data to determine users' movements between staying in a place, walking and driving. By using boosted logistic regression in two phases, an accuracy of 80.85\% is reached for walking and driving. Although collecting GSM data is essentially a network-centric approach, data collection in this study mainly relied on limited number of lab members with a designed software for recording GSM records. Thus, the advantages of network-centric approaches are not fully in effect in this study. Wang~\textit{et al.}~\cite{wang2010transportation} used coarse-grained unlabelled call detail records to infer transportation mode between pairs of defined origins and destinations. K-means algorithm was used in this study to detect mode of transportation. Low accuracy in detecting users' location, ping-pong handover effect, and privacy issues are mentioned in this study as drawbacks of using GSM data. It should be noted that the GSM data is not readily available and needs cooperation from cellular network providers. Mun~\textit{et al.}~\cite{mun2008parsimonious} coupled Wi-Fi and GSM data to reach a classification accuracy of 88\% for walking and driving in urban areas. They used decision tree to differentiate between walking, driving, and dwelling. Features used for classification in this experiment include Wi-Fi signal strength variance, duration of the dominant Wi-Fi access point, number of cell IDs that device connects to and residence time in cell footprint. Similar to Sohn's study, data collection in this study was designed as a controlled procedure using limited number of participants actively involved in the experiment. Unlike GPS data, for which accuracy is significantly affected in high-rise urban environments, Wi-Fi data can be implemented even in indoor environments. Krumm and Horvitz~\cite{krumm2004locadio} used Wi-Fi signal strengths and their variance as inputs to a Hidden Markov Model for smoothing transitions between the inferred states of still and moving in indoor environments. An accuracy of 87\% was achieved in this study. The indoor environments are more controlled and have lesser heterogeneity, compared to outdoor environments. This study also did not consider transportation modes other than walking. Beaulieu and Farooq~\cite{beaulieu2019} used Wi-Fi data collected on a 14 blocks pedestrianized street in Montreal to develop the next location choice model. They developed a dynamic mixed logit model with agent effects to achieve a maximum prediction performance of 70\%.

Growing interest in deep neural networks has led researchers to investigate deep architectures for the purpose of mode detection. Endo~\textit{et al.}~\cite{endo2016deep} developed a deep neural network with fully connected layers to extract high-level features. Image-based deep features were combined with manual
features, and used as the input for a traditional classifier. Despite successfully implementing a deep network, the study lacks motion features such as speed and acceleration.
In another study, Wang~\textit{et al.}~\cite{wang2017detecting} defined point features as a time series of speed, headway change, time interval, and distance between the GPS points. These features were then combined with manual features, and fed to a deep neural network. Maenpaa~\textit{et al.}~\cite{maenpaa2017travel} used Random Forest, Neural networks, and Bayes classifier to identify mode of transportation. In their study, they tried to generate and test new features and find the most relevant features for identifying transportation mode. The authors found that random forest performed better when compared to the neural networks. Despite using high-level features and developing deep neural networks, the accuracies obtained using these algorithms are still lower than some prediction accuracies obtained by simpler machine learning algorithms. However, more advanced neural networks were not tested and analyzed in the early applications of neural networks for mode detection. Recently, Dabiri and Heaslip~\cite{dabiri2018inferring}, and Yazdizadeh~\textit{et al.}~\cite{yazdizadeh2019ensemble} developed Convolutional Neural Network (CNN) architectures with different types of layers that are fed with an input layer with kinematic characteristics. They achieved a total accuracy of 84.8\% and 91.8\% respectively. CNN architecture is well suited for user-centric approaches where the location is regularly sampled in terms of time. However, such an architecture is not suited for network-centric approaches where the user is observed at fixed locations that may or may not overlap. 

In spite of promising performances of using deep neural networks for mode detection, all these studies are concentrated on labelled data from user-centric approaches, which requires vast amounts of labelled data. Semi-supervised learning is one of the possible approaches to address this issue by utilizing unlabelled data in order to enhance hypotheses obtained from labelled data~\cite{zhu05survey}. In short, semi-supervised algorithms add unlabelled samples to the training data and the classifier is retrained on the new augmented training dataset. In one of the few studies in using semi-supervised deep learning approaches for mode detection, Dabiri~\textit{el al.}~\cite{dabiri2019semi} proposed a semi-supervised Convolutional Autoencoder architecture to predict transportation mode using GPS data. They achieved an accuracy of 76.8\% with 4 convolutional layer, showing a decrease in accuracy in higher number of layers. In another study, Yazdizadeh~\textit{et al.}~\cite{yazdizadeh2019semi} developed a Deep Convolutional Generative Adversarial Network (DCGAN) in a semi-supervised manner. Generator layers were used in this study to create fake samples of GPS trajectory. Fake samples along with real samples from \textit{MTL traject} dataset were then fed into the discriminator layer as the input layer. Using this architecture, a total accuracy of 83.4\% was achieved.

\cref{tab:lit} presents a brief summary of the discussed literature. In short, studies using user-centred approaches provide higher accuracy in prediction, while they require involvement of participants which may lead the model to be biased towards certain type of users. Moreover, large scale implementation of such models may be limited due to unavailability of such information from the population. On the other hand, network-centric approaches exhibit lower positioning accuracy, and thus result in fairly lower prediction accuracy. However, such approaches passively collect data from network users, leading to more realistic inference from the network, which can be implemented in a scalable as well as economical way.


The advantages of Wi-Fi networks over other sources in terms of their granularity and ubiquity, along with the promising performance gains of deep residual networks in recent years, made us explore the feasibility of using Wi-Fi signals as the only source of data to detect transportation mode. Moreover, one of the drawbacks of using Wi-Fi networks for mode detection is their relatively small size of collected data in comparison to GPS-based methods. To overcome the difficulty of labelling in network-based approaches, and address the small size of data, we introduced a semi-supervised version of the model that incorporates unlabelled data. This paper, is a continuation of our previous study~\cite{kalatian2018mobility}, in which only labelled data were incorporated to train a deep multi-layer perceptron network. By collecting more data, incorporating unlabelled data, and developing more advanced semi-supervised ResNet architecture, this paper tries to address the same problem of detecting transportation mode, using a completely different approach that avoids overfitting and low sample size issues.


\begin{table*}[t]
\centering

\caption{A summary of relevant research}
\label{tab:lit}
\scalebox{0.9}{
\small\addtolength{\tabcolsep}{-4pt}
\begin{tabular}{|l|l|l|l|l|p{38mm}|l|}
\hline

\textbf{}                                      & \textbf{Data}                                          & \textbf{Method}                                               & \textbf{Study}                                     & \textbf{Acc(\%)} & \textbf{Advantages}                                                                                                                      & \textbf{Limitations}                                                                                         \\ \hline\hline
{\multirow{5}{*}[-0.5em]{{\rotatebox[origin=c]{90}{\textbf{Network-centric}}}}} & \multirow{2}{*}{Wi-Fi}                                        & HMM                                                           & \cite{krumm2004locadio}           & 87                & Passive data collection                                                                                                                  &  Indoor environment                                                                            \\ \cline{3-7} 
                            &                                                               & \begin{tabular}[c]{@{}l@{}}Mixed\\ logit\end{tabular}                                                    & \cite{beaulieu2019}               & 70                & Predicting next location                                                                                                                 & Low accuracy                                                                                                 \\ \cline{2-7} 
                                 & \multirow{2}{*}[1em]{GSM}                                          & \begin{tabular}[c]{@{}l@{}}Logistic\\ Regression\end{tabular} & \cite{sohn2006mobility}           & 80.85             & \begin{tabular}[c]{@{}l@{}l@{}}Coarse-grained data\\ can better address\\ privacy concerns\end{tabular}                                       & \begin{tabular}[c]{@{}l@{}}Tested only on limited \\ number of lab members\end{tabular}                      \\ \cline{3-7} 
                                 &                                                               & K-means                                                       & \cite{wang2010transportation}     & NA                & Passively collected large data                                                                                                           & \begin{tabular}[c]{@{}l@{}l@{}}Validation is not possible, \\ difficult to apply\\ to congested areas\end{tabular} \\ \cline{2-7} 
                                 &
                                 \begin{tabular}[c]{@{}l@{}}Wi-Fi+\\ GSM\end{tabular}
                                 & \begin{tabular}[c]{@{}l@{}}Decision\\Tree\end{tabular}                                                     & \cite{mun2008parsimonious}        & 88                & 
                                 \begin{tabular}[c]{@{}l@{}}Enhanced performance by\\ using two sensor types\end{tabular} 
                                 & \begin{tabular}[c]{@{}l@{}}Tested only on limited\\ number of lab members\end{tabular}                       \\ \hline
\multirow{13}{*}[-2em]{\multirow{5}{*}{{\rotatebox[origin=c]{90}{\textbf{User-centric}}}}}                        & \multirow{11}{*}[6em]{GPS}                                         & \begin{tabular}[c]{@{}l@{}}Graph\\ based\end{tabular}                                                    & \cite{zheng2008understanding}     & 76.2              & \begin{tabular}[c]{@{}l@{}l@{}}Considers both real world\\ constraints  and typical user \\behaviour\end{tabular}                              & Low accuracy                                                                                                 \\ \cline{3-7} 
                                                       &                                                               & \multirow{4}{*}[1em]{\begin{tabular}[c]{@{}l@{}}Decision\\Tree\end{tabular} }                                     & \cite{xiao2017identifying}        & 93.19             & High prediction accuracy                                                                                                                 & Complex preprocessing                                                                                        \\ \cline{4-7} 
                                                       &                                                               &                                                               & \cite{maenpaa2017travel}          & 90.7              & New derived features                                                                                                         & \begin{tabular}[c]{@{}l@{}}Not tested on\\ advanced networks\end{tabular}                                                                             \\ \cline{4-7} 
                                                       &                                                               &                                                               & \cite{stenneth2011transportation} & 93.7              &                                       \begin{tabular}[c]{@{}l@{}}Using network information\\  to boost performance \end{tabular}                                                  & \begin{tabular}[c]{@{}l@{}}Network information may \\ not be available\end{tabular}                          \\ \cline{4-7} 
                                                       &                                                               &                                                               & \cite{yazdizadeh2019automated}    & 87                & \begin{tabular}[c]{@{}l@{}}Using network and \\ social media information\end{tabular}                        & Participant involvement                                                                              \\ \cline{3-7} 
                                                       &                                                               & \multirow{2}{*}{DNN}                                          & \cite{endo2016deep}               & 83.2              & Higher level features                                                                                                           & Lack of motion features                                                                                      \\ \cline{4-7} 
                                                       &                                                               &                                                               & \cite{wang2017detecting}          & 74.1              & No hand-crafted features                                                                                                        & \begin{tabular}[c]{@{}l@{}}Not tested on\\ advanced networks\end{tabular}                                                                               \\ \cline{3-7} 
                                                       &                                                               & \multirow{3}{*}{CNN}                                          & \cite{dabiri2018inferring}        & 84.8              & No hand-crafted features                                                                                                         & Participant involvement                                                                              \\ \cline{4-7} 
                                                       &                                                               &                                                               & \cite{dabiri2019semi}             & 76.8              &        \begin{tabular}[c]{@{}l@{}}No hand-crafted features ,  \\ semi-supervised  \end{tabular}                                                                                & Participant involvement                                                                              \\ \cline{4-7} 
                                                       &                                                               &                                                               & \cite{yazdizadeh2019ensemble}     & 91.8              & \begin{tabular}[c]{@{}l@{}}Using Random Forest\\ as meta learner\end{tabular}                          & Participant involvement                                                                              \\ \cline{3-7} 
                                                       &                                                               & GAN                                                           & \cite{yazdizadeh2019semi}         & 83.4              & \begin{tabular}[c]{@{}l@{}}Semi-supervised, using GAN\\  in none-image context\end{tabular}                                       & \begin{tabular}[c]{@{}l@{}}Participant involvement ,\\ low accuracy\end{tabular}                     \\ \cline{2-7} 
                                                       & 
                                                       \multirow{3}{*}[1em]{Hybrid}& \begin{tabular}[c]{@{}l@{}}HMM +\\ DT\end{tabular}            & \cite{reddy2008determining}       & 90                &   \begin{tabular}[c]{@{}l@{}}No preprocessing required,  \\ high accuracy  \end{tabular}                                                                                                 & Very limited participants                                                                                    \\ \cline{3-7} 
                                                       &                                                         & HMM                                                           & \cite{bantis2017you}              & 78                & \begin{tabular}[c]{@{}l@{}l@{}}Personal characteristics, \\ Complementary to\\ traditional surveys\end{tabular} & \begin{tabular}[c]{@{}l@{}}Demographic information\\ required
                                 \end{tabular}                          \\
                                                       \cline{3-7} 
                                                       &                                                         & \begin{tabular}[c]{@{}l@{}}RF\\ GBM\end{tabular}                                                         & \cite{efthymiou2019transportation}              & 85.5                & \begin{tabular}[c]{@{}l@{}}Reduced battery\\ consumption\end{tabular} & Participant involvement 
                                                                                                                               \\ \hline
\end{tabular}
}
\end{table*}