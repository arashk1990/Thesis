% Packages

% WiFi springer packages
\usepackage{multirow,multicol}
\usepackage{graphicx}
\usepackage{subcaption}
\usepackage{rotating}
\usepackage{tikz}

\usepackage{tabularx}
\newcolumntype{b}{>{\hsize=.7\hsize}X}
\newcolumntype{s}{>{\hsize=.3\hsize}X}
%% The amssymb package provides various useful mathematical symbols
\usepackage{amssymb}
%% The amsthm package provides extended theorem environments


\usetikzlibrary{positioning}
\usepackage{xcolor}
\usepackage{array,calc}
\newlength\Wlena \newlength\Wlenb \newlength\Wlenc \newlength\Wlend
\settowidth{\Wlenb}{2-way \& median}
\setlength{\Wlena}{\dimexpr0.5\Wlenb-\tabcolsep-\arrayrulewidth/2\relax}
\setlength{\Wlenc}{\dimexpr1.5\Wlenb+\tabcolsep+\arrayrulewidth/2\relax}
\setlength{\Wlend}{\dimexpr0.5\Wlenb+\tabcolsep+\arrayrulewidth/2\relax}
\newcolumntype{P}[1]{>{\centering\arraybackslash}p{#1}} % centered "p" column
\newcommand\Wmcii[1]{\multicolumn{8}{P{\Wlend}|}{#1}}  % handy shortcut macros
\newcommand\Wmciii[1]{\multicolumn{16}{P{\Wlenc}|}{#1}}
%%%
%%%
%% VR wait

\usepackage{booktabs}

%% The amssymb package provides various useful mathematical symbols
\usepackage{amssymb}
\usepackage{longtable}
%% The amsthm package provides extended theorem environments
%\usepackage[linesnumbered,ruled]{algorithm2e}

%% The lineno packages adds line numbers. Start line numbering with
%% \begin{linenumbers}, end it with \end{linenumbers}. Or switch it on
%% for the whole article with \linenumbers after \end{frontmatter}.
\usepackage{lineno}
\usepackage{array,calc}
\usepackage{lscape}
\usepackage{float}
\floatstyle{plaintop}
\restylefloat{table}
\usetikzlibrary{positioning}
\newlength\lena \newlength\lenb \newlength\lenc \newlength\lend
\settowidth{\lenb}{2-way with median}
\setlength{\lena}{\dimexpr0.5\lenb-\tabcolsep-\arrayrulewidth/2\relax}
\setlength{\lenc}{\dimexpr1.5\lenb+\tabcolsep+\arrayrulewidth/2\relax}
\setlength{\lend}{\dimexpr0.5\lenb+\tabcolsep+\arrayrulewidth/2\relax}
%\newcolumntype{P}[1]{>{\centering\arraybackslash}p{#1}} % centered "p" column
\newcommand\mcii[1]{\multicolumn{10}{P{\lenb}|}{#1}}  % handy shortcut macros
\newcommand\mciii[1]{\multicolumn{15}{P{\lenc}|}{#1}}
\newcommand\mciiiii[1]{\multicolumn{6}{P{\lend}|}{#1}}
\usepackage{adjustbox}

\usepackage{array,calc}
\usepackage{makecell}
\newcommand{\note}[1]{
\begin{boxitpara}{}%
\textbf{\underline{Internal note}:} #1
\end{boxitpara}}

%Traj
\newsavebox{\measurebox}




\usepackage[utf8]{inputenc}
\usepackage[T1]{fontenc}
\usepackage[backend=biber,sorting=none,natbib=true,giveninits=true,maxbibnames=99,maxnames=2]{biblatex}
%\usepackage[dvips]{graphicx}
\usepackage{appendix}
\usepackage{colortbl}
\usepackage{amsmath,amssymb,amsthm}
\usepackage{textcomp}
\usepackage{hyperref}
\usepackage{tabu}
\usepackage{listings}
\usepackage{pdfpages}
\usepackage{tocloft}
\usepackage[ruled,vlined]{algorithm2e}
\usepackage[nameinlink,capitalize]{cleveref}

% package configurations
\hypersetup{
    breaklinks=true,colorlinks=true,
    linkcolor=blue,urlcolor=blue,citecolor=blue
}

\captionsetup{%
    % justification=raggedright,
    justification=justified,
    singlelinecheck=off,
}

\addbibresource{bibliography.bib}

\newtheorem{exm}{Example}

\newtheorem{lem}{Lemma}

%%%%%%%%%%%%%%%%%%%%%%%%%%%%%%%%%%%%%%%
%% left, right margins and textwidth %%
%%%%%%%%%%%%%%%%%%%%%%%%%%%%%%%%%%%%%%%
\setlrmarginsandblock{1.5in}{1.5in}{*}

%% for main body, bottom of text at 1in, footer below%% top of header at 1in, first text line double spaced%% below base of header
\newlength{\linespace}
\setlength{\linespace}{\baselineskip} % current equivalent of \onelineskip
\setlength{\headheight}{\onelineskip}
\setlength{\headsep}{\linespace}
% \addtolength{\headsep}{-\topskip}
% \setlength{\uppermargin}{1in}
% \addtolength{\uppermargin}{\headheight}
% \addtolength{\uppermargin}{\headsep}

%% and for the bottom
% \setlength{\lowermargin}{1in}
% \setlength{\textheight}{\paperheight}
% \addtolength{\textheight}{-\uppermargin}
% \addtolength{\textheight}{-\lowermargin}

%% footnote settings
% \setlength{\footskip}{\onelineskip}
% \setlength{\footnotesep}{\onelineskip}

%% the fiddle lengths (..ta.. for title/approval page, others for prelims)
% \newlength{\toptafiddle} \setlength{\toptafiddle}{2\linespace}
% \newlength{\bottafiddle} \setlength{\bottafiddle}{0pt}
% \newlength{\topfiddle}   \setlength{\topfiddle}{\toptafiddle}
% \newlength{\botfiddle}   \setlength{\botfiddle}{\onelineskip}

\setlength{\parindent}{2em}
\checkandfixthelayout[nearest]

\renewcommand*{\cftchapterfont}{\normalfont}
\renewcommand*{\cftchapterpagefont}{\normalfont}
\renewcommand*{\cftchapterleader}{%
    \cftchapterfont\cftdotfill{\cftchapterdotsep}}
\renewcommand*{\cftchapterdotsep}{\cftdotsep}

%%% no extra space before the entry, or in the LoF/LoT
\setlength{\cftbeforechapterskip}{0pt plus 0pt}
\renewcommand*{\insertchapterspace}{}

%%%%%%%%%%%%%%%%%
%% Back matter %%
%%%%%%%%%%%%%%%%%

% \renewcommand*{\bibname}{References}
% \setlength{\bibitemsep}{\onelineskip}
% \renewcommand*{\biblistextra}{%
%     \setlength{\itemsep}{\bibitemsep}
%     \setlength{\labelwidth}{0pt}
%     \setlength{\leftmargin}{3em}% hanging indent
%     \setlength{\itemindent}{-\leftmargin}}

%% endnotes
\renewcommand*{\notesname}{Notes}
\renewcommand*{\notedivision}{%
    \AsuSpacing\chapter*{\notesname}
    \addcontentsline{toc}{chapter}{\notesname}
    \SingleSpacing}

%%%%%%%%%%%%%%%%
%% Title Page %%
%%%%%%%%%%%%%%%%
\setlength{\droptitle}{20pt}
\newcommand{\statement}[1]{%
\gdef\puB{#1}}\newcommand{\puB}{}
\renewcommand{\maketitlehookd}{%
    \vspace*{9em}\centering \puB}

%%%%%%%%%%%%%%%%%%%%%%%%%%%%
%% (subsub)section styles %%
%%%%%%%%%%%%%%%%%%%%%%%%%%%%
\setsecnumdepth{subsection}
\maxtocdepth{subsection}

%%%%%%%%%%%%%%
% TOC styles %
%%%%%%%%%%%%%%

%%%%%%%%%%%%%%%%%
%% page styles %%
%%%%%%%%%%%%%%%%%
\pagestyle{ruled}
\makeoddfoot{plain}{}{\thepage}{}
\makeevenfoot{plain}{}{\thepage}{}
\makeoddfoot{ruled}{}{\thepage}{}
\makeevenfoot{ruled}{}{\thepage}{}
\OnehalfSpacing
