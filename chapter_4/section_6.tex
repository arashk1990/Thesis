\section{Discussion and policy implications}
\label{S:discuss}
There are several key practical implications and useful policy recommendations that can be derived from the virtual reality experiments, models, and results of this study, which can be of use to urban planners, policy makers, manufacturers of automated vehicles, and researchers seeking to implement virtual reality and data-driven modelling in their work.

During the VR data collection campaign, out of 180 participants, a total of 113 adults participated who were able to complete at least one of their 30 assigned scenarios. However, after data cleaning, 2,291 responses remained which account for 67.6\% of our expectation on the number of total responses (113 $\times$ 30). Aside from dangerous crossings with PETs less than 1.5 seconds that we removed from the data for the purpose of the study, two major reasons contribute to this inefficiency in data collection were:
\begin{itemize}
    \item Participants feeling motion sickness after some attempts: One of the major reasons of failing to use a participants' task data was motion sickness. Despite priorly asking of participants' background health issues, not all the participants were aware of their reaction in VR. Widespread development of VR can help ease this issue in future studies with participants better knowing their reactions and technology building smoother environments.
    \item Participants not following guidelines: Despite the training session dedicated to teach participants on how to do the tasks, some experiments were removed due to their failure to do so. Easier to follow guidelines and continues reminders during the experiment can help to prevent this issue in the future. 
\end{itemize}

Regarding participants' experience in VR, extra attention should be paid to children and senior participants in terms of training and educational programs, and their comfort during the experiments. Moreover, providing facilities for people with special needs should be taken into account in future research to make experiments more inclusive. Out of the 180 participants recruited for the experiments, 20 who were children and senior participants, failed to complete any of their tasks. Although we had one person using wheelchair as a participant, more accessibility services should be provided in future to increase the participation of people with different disabilities. As per our observations, some children tended to see the virtual reality tools as gaming platforms. Irrational crossing behaviours to explore the 3D environment, followed by their loss of interest in continuing the experiment after observation of different elements of the environment, led to their failure in completing the experiment tasks. The virtual vehicles were designed such that they could foresee if the pedestrian is in danger and would apply braking when needed. Some teenage participants realized that and started to play with the virtual vehicles. They would come in front of an vehicles so it would stop. Participant would then move back, and virtual vehicle would sense this action and start moving, at which point, the participant would move in front of the vehicle again causing it to stop, repeating such actions several times. This behaviour is expected to happen when the fully automated vehicles are operating on urban roads, especially in lower speeds, resulting in disruption of the upstream traffic. To avoid such behaviour, VR based education and training is needed. Municipalities can also look into proactive regulations to minimize such disruptions. As for senior participants, their inconvenience while using VR headsets, inability to ambulate independently without mobility devices such as wheelchair, and feeling nauseated and fatigued while interacting with 3D immersive environment were identified in our observations as the main cause of them not completing the tasks.

Interpretation of our model results can be useful for practical implications for urban decision makers and car manufacturers. Our results show that participants tend to be more conservative and cautious in the presence of automated vehicles. Either in fully automated or mixed automated and human-driven conditions of traffic, participants waited longer compared to solely human-driven conditions. Longer wait time in the presence of AVs is in particular bolder for participants aged over 50, and in congested areas with higher traffic densities. Longer wait before crossing an environment with automated vehicles in it is intuitive, and was expected considering the unfamiliarity of pedestrian to vehicles with no drivers. This is of importance considering that shorter waiting times in safe crossings of pedestrians imply more trust and confidence confronting the vehicles. Thus, predictive models for pedestrian wait time should take into account the differences of pedestrian behaviour facing vehicles with no drivers. Also, considering the unfamiliarity of pedestrians with automated vehicles in the first years of introduction of AVs, this observation requires attention of city planners and decision makers to make modifications needed to enhance pedestrians' crossing experience. Demography of the area, traffic parameters, and road geometry should all be considered in the modifications to be made. Nationwide educational training programs should be practiced before the transition to automated environments, to familiarize pedestrians with new dynamics of the city. Immersive and dynamic virtual reality technology can play an important role in such programs, to ensure providing safe and naturalistic experience to users. Manufacturers should consider alternative ways to improve the communications and quality of interaction between the driver and pedestrians in automated environments. Some manufacturers, like drive.ai, have introduced screens on their vehicles that can give visual cues to pedestrians. Similar to \cite{fridman2017walk}, \cite{chang2017eyes}, \cite{de2019external}, etc., such new treatments can be systematically optimized and their effectiveness can be studied in VIRE environment.
Based on results of our study, narrower lane widths, lower traffic densities, and better sight distances are also revealed to be affecting parameters in pedestrians' crossing behaviors, leading to shorter wait times. To the best of our knowledge, the effect of road design parameters in the context of automated environments, i.e. lane width and existence of median, are not investigated in the literature despite their importance in crossing behaviour of pedestrians in traditional studies. Several studies both in the automated environments~\citep{rasouli2019autonomous} and human-driven environments~\citep{sun2015estimation} have investigated the effect of environmental variable, with which our results are in line. The effect of traffic parameters, in various forms, has also been explored in traditional studies of pedestrians~\citep{schmidt2009pedestrians,ishaque2008behavioural}, and we show the same pattern exists in the new context. Wider and more comfortable sidewalks, narrower lane widths, enhanced lighting equipment, and incorporation of pedestrian-to-vehicle communication technologies are some of the solutions that can be implemented before diving into future automated urban areas. Our study also reveals that having frequent walking habits positively affects the crossing experience. In the context of automated vehicles, several studies had addressed cultural differences as a contributing factor to pedestrians crossing behaviour. While this observation might hold true, using more individualized parameters like walking habits might help better address the cultural differences, particularly in cities with diverse population. Promoting active modes and developing more pedestrian-friendly infrastructures could lead to better crossing experience on the streets of the future. Multi-level analysis in our study also showed that having walking habits leads to shorter waiting times even among groups that are conventionally known to be more conservative, i.e. females and elderly people. Gender alone did not appear to have a significant contribution in our study, except for in automated and mixed traffic conditions, one-way roads, and among participants who use car as their main mode of transportation. Gender has traditionally been a significant variable in pedestrians' crossing behaviour. The limited contribution of gender in our study might be due to the virtual nature of our experiments, in which perception of risk is inevitably lower that real roads. Our results also showed that in poor sight distances, as well as among female participants, participants aged over 50 tend to have longer waiting times, which reveals the necessity of a special attention to this group.