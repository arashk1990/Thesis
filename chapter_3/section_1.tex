\section{Introduction}
\label{S:Dint}
Pedestrian crossing behaviour is a topic of interest as it gives insights into traffic lights design, pedestrian safety, roadway layouts design and traffic flow optimization. This study aims at exploring one of the attributes of pedestrian street-crossing behaviour: waiting time. Research on pedestrian waiting time analysis has gained popularity as pedestrian-vehicle accidents result in a large proportion of total accident deaths. Pedestrian violations, i.e. J-waling, disobeying pedestrian lights and failure to yield to vehicles, have been a major source of traffic injury and fatality in recent years~\cite{Canada2005,Canada2010,Canada2015}. Despite implementation of various safety measures, widespread educational programs, etc., rate of pedestrian-related fatalities and injuries in Canada has increased in the last decade. According to the Canadian Motor Vehicle Traffic Collision Statistics, proportion of fatal accidents involving pedestrians to all fatal accidents has increased from 11.8\% in 2005 to 15.2\% in 2015~\cite{Canada2005,Canada2015}. A reason for this increase can be traced back to the rise of smartphones and their applications in everyday life. Pedestrians are becoming more distracted in recent years, using their phones for talking, texting, surfing the web, looking for directions, or playing games~\cite{Canada2010}.

In order for a pedestrian to cross an unsignalized intersection, individuals should wait for a gap that, based on each pedestrian’s abilities, allows safe crossing of the street. An individual waiting to cross an unsignalized intersection is required to concentrate and evaluate whether each gap satisfies the spatial and temporal requirements of a safe cross. Pedestrian waiting time has a significant impact on unsafe crossings. Studies suggest a positive correlation between waiting time before initiating a cross and the violations caused by pedestrians~\cite{brosseau2013impact}. Using mobile phones and the distraction caused by them negatively affects pedestrians’ ability to cross and thus, increases the pedestrian-related accident rates~\cite{banducci2016effects,dey2017impact}. By implementing safety solutions, waiting time can be affected for distracted pedestrian.

This paper investigates the effect of smartphone distraction on pedestrians’ waiting time by adapting a head mounted immersive virtual reality environment. In conventional field experiments on pedestrian behaviours, it is often difficult to implement different scenarios as the results may be disastrous in terms of participants’ safety. In addition, it may be expensive or unsafe to repeat an experiment with the exact same traffic conditions to capture the effects of an implemented safety measure. To overcome such problems, pictures, videos and photomontages can be used to assess participants perception. With the development of interactive computer-generated experiences, Virtual Reality (VR) experiments have gained popularity in various research fields. Using a head mounted VR display device, virtual scenes can be directly projected to the participants so that the they will be immersed in the simulated environment~\cite{farooq2018virtual}. Scenarios used in experiments may be unsafe or expensive to apply on real roads, due to reasons such as dangerous implementations or lack of infrastructures. VR simulator allows running such scenarios, along with scenarios containing new technologies or services that participants have limited mental image of.

Data collection procedure for this study involves a virtual immersive reality environment (VIRE). To investigate distracted pedestrians’ waiting time before crossing an unsignalized intersection, we asked participants to cross a simulated unsignalized intersection in VIRE in three different simulation tasks: Task 1. no distraction, Task 2. distraction by a maze on a smartphone, and Task 3. distraction by a maze game on a smartphone with flashing LED lights installed on the crosswalk as a safety encounter. After data collection and conducting a descriptive analysis of the data, a cox proportional hazards model is adopted to further explore the effects of different sociodemographic and traffic parameters on pedestrians’ waiting times.

This paper is organized as follows: in \cref{S:Dback}, we review the previous works on the subject. In \cref{S:D3} data collection procedure is described. \cref{S:D4} discusses the model structure. Moreover, variables derived from data and a descriptive analysis on them are then explained in \cref{S:D5}. Proportional Hazards model and its implementation on the data is then elaborated in \cref{S:D6}. In the end, conclusions and future research plans are covered in \cref{S:D7}.

