\part{Conclusion}

\chapter{Summary}
\label{chap7}
This dissertation provides a series of interconnected research studies that address pedestrian issues in smart cities through a lens of modern data sources and the new data-driven paradigm of machine learning. By introducing alternative ways of data collection, data-driven approaches for analyzing the collected data, detailed interpretability methods for explaining the models, and application of the frameworks to a series of futuristic scenarios, this work endeavours to provide solutions to the challenges and uncertainties involved in the path toward smart cities. \cref{chap1b} presents the background of pedestrian research in two directions. First, alternative data sources to traditional counts, interviews, revealed and stated preference surveys, etc. were discussed. Location-aware sensors and virtual reality tools, and automated vehicle sensor suites were explored as novel alternative resources for data collection. In the second part, studies that concern the interactions of pedestrians and vehicles were reviewed. The methodological history of pedestrian crossing behaviour, with a focus on wait time and trajectory, were provided in this chapter, as well as the new trends concerning the use of new data-driven machine learning models to analyze pedestrian behaviour. The chapter highlights the strengths and weaknesses of current and proposed methods of pedestrian crossing behaviour modelling. To overcome the interpretability challenges of machine learning models and add a descriptive perspective to the predictive perspective of these models, recent trends and methodologies in machine learning interpretability were discussed in this chapter.

\cref{chap2} is based on the first research work of this thesis, demonstrating the use of activity logs from ubiquitous Wi-Fi networks as a novel data source enabling large-scale passive collection in congested urban areas. By developing a deep residual neural network in a semi-supervised manner, the mode of transportation of the network users can be detected in highly congested urban areas, enabling tracking and counting pedestrians. The semi-supervised nature of the algorithm helps to improve the performance of the model by using unlabelled, easily-collected, low-cost Wi-Fi data. Among the three modes of transportation investigated, our proposed framework showed a promising performance of 81.4\% in the precise detection of pedestrians.

The second work concerns the use of virtual reality, another novel source of data for conducting controlled experiments or an alternative to stated preferences surveys. In \cref{chap3}, VIRE~\cite{farooqvire}, an interactive virtual reality framework, was utilized to analyze the distracting effect of smartphones on crossing pedestrians. The survival analysis proposed in this study showed that an increase in
initial walk speed, engagement with smartphones during crossings, bigger minimum
missed gaps and unsafe crossings lead to shorter waiting times of pedestrians. On the other hand, distraction by smartphones during the waiting time meant
longer waiting times for the participants in our experiment. 

The third work focuses on the interactions of pedestrians and automated vehicles and the policy suggestions and practices recommendations toward the transition to fully automated environments. \cref{chap4} introduced a large-scale data collection campaign using VIRE, during which the crossing behaviours of pedestrians were observed and assessed in the presence of automated vehicles. By developing neural network-based survival models, this chapter seeks to propose a data-driven version of a well-known traditional model. A comprehensive interpretability analysis of the proposed models was also included in this chapter to compensate for the \textit{black box} nature of neural networks as the major barrier for their extensive use in some areas. Based on the interpretations of the results of our framework, we suggested educational programs for children, enhanced safety measures for seniors, promotion of active modes of transportation, and revised traffic rules and regulations to have a move toward pedestrian-friendly urban areas.

Furthermore, \cref{chap6} tackles the same problem of pedestrian crossing behaviour in automated environments with a focus on trajectory prediction. This chapter highlights how open-source video datasets of available test AVs still lack the essential information for pedestrian studies in the context of transportation engineering. By selecting variables from VIRE that can be captured by a hypothetical automated vehicle, this chapter provides insight into the required data for pedestrian trajectory prediction. We showed that by the incorporating contextual information from the environment, a decrease in prediction error and overfitting of the model could be achieved. Investigating the interpretation of the results, we suggested that in order to have a pedestrian-oriented automated vehicle dataset, diverse weather and vision conditions, as well as different traffic scenarios need to be included in the data collection process. 

\section{Research contributions}
\label{chap7:sec1}
This dissertation is a collection of four research articles, each making an original contribution to the state-of-the-art  research on pedestrian dynamics. The following summarizes the main contributions of the articles from \cref{part_2:dissertation_articles} of the thesis:

\subsubsection*{Semi-supervised approach to mode detection in Wi-Fi signals}
The third chapter of this dissertation started with the idea of using labelled and unlabelled Wi-Fi logs data as an alternative to traditional travel surveys, video tracking, and recently popular GPS data.
We developed a novel framework by incorporating a very deep residual network, trained using pseudo-labelling, which enabled using the collected unlabelled Wi-Fi data and captured the high levels of nonlinearity within the data.
This approach is useful as it collects and utilizes data from the network users without their involvement, and even when they are not directly part of the experiment. This aspect of our approach overcomes the constraints of the GPS based approach that requires the involvement of participants by installing an app and also labelling their modes.
The proposed framework is an application of machine learning techniques that has the potential to benefit from ubiquitous communication networks to improve smart transportation planning, operations, and design. One advantage of the proposed framework is its transferability to data from other location-aware technologies, with minimum modifications required. 
Infrastructures for collecting Bluetooth and Wi-Fi signals are already implemented in Toronto and several other cities in the world. Furthermore, 5G networks are forthcoming technologies that, similar to Wi-Fi networks, have a higher antenna density and thus strong potential for precise localization. The data generated from such sources can be retrieved from the municipalities and telecommunication companies without the need for any extra facility. By harnessing the data that smart cities have at their disposal, automated solutions for tasks such as detailed mode-specific counts, origin-destination matrices, travel time estimations, congestion prediction, etc. can be deployed. The anonymized nature of the collected Wi-Fi data also addresses concerns about infringing on the privacy rights of users. 


\subsubsection*{Distracted pedestrian behaviour analysis in immersive virtual reality}
The fourth chapter utilized VIRE, a virtual reality immersive environment that enables collecting data in a safe and controlled environment~\cite{farooqvire}. VIRE was proposed as a more realistic alternative to state preference surveys, which enables collecting data for scenarios that are not easy to implement in real-world settings. As it may be unsafe to track the behaviour of pedestrians while they are distracted using their phones, VIRE appears to be an alternative with minimal risk and great potential. With the extensive ability of smartphones to engage users, pedestrians have become more distracted in the past decade, leading to a rise in pedestrian-related accident rates. By applying a survival analysis model in this chapter, the effect of this distraction on pedestrian crossing behaviour was investigated. The results of this study showed that engagement with smartphones leads to shorter waiting times of pedestrians, but using LED treatment lights on the crosswalk as a safety measure can smooth this impact.

\subsubsection*{Pedestrian and automated vehicle interactions: Wait time behaviour analysis}
The fifth chapter began a deep investigation of the interactions of pedestrians with automated vehicles. By focusing on unsignalized mid-block crossings, the waiting time of pedestrians before initiating a cross was analyzed in this chapter. An enhanced version of survival models was used to understand the effects of various covariates on the wait time, followed by a post-hoc game theory interpretability model. This chapter contributes to the methodology of survival analysis by developing a data-driven version of the Cox Proportional Hazards model, empowered with neural networks to capture higher levels of nonlinearities among the data. To collect high-quality data for this research study, a large data collection campaign was conducted with over 180 participants with a various demographics in a diverse participant pool. To the best of our knowledge, this was the most comprehensive virtual reality data collection campaign in the transportation field, which can pave the way for the adoption of VR-based studies in the transportation community. We provided insights into challenges that other researchers interested in large-scale VR data collections might face in their experiments and discussed practical recommendations to follow prior to or during such data collection campaigns. Another contribution of this chapter to the transportation research community is the recommendation of policy suggestions and practical implications for pedestrian crossing behaviour in the presence of automated vehicles. By exploring the interpretation of our model, we studied the effect of various parameters from various points of view, including policy-makers, city-planners, road users, and Original Equipment Manufacturers (OEM) perspectives.


\subsubsection*{Pedestrian and automated vehicle interactions: Movement behaviour analysis}
In the sixth chapter, data-driven time-series data was developed to utilize the data from initial movements of pedestrians as well as contextual information of the environment to predict the trajectory of pedestrians. This methodological framework allows for using the information of the crossing environment as auxiliary information and enhances the performance of the models based on them. We showed that incorporation of contextual information in the model reduces the overfitting of the model to the training data. Contextual information that was taken into account in this research included road specifications, i.e. lane width and type of road, traffic parameters, i.e. speed limit, arrival rate, and environmental conditions, i.e. weather conditions and time of the day. By choosing the contextual information in a way that can be obtained by AVs, suggestions were provided for future data releases of automated vehicles to account for pedestrian-oriented points of view.

\section{Limitations}
\label{chap7:sec2}
Every analysis approach has its own set of advantages and drawbacks, and it is the analyst's job to be aware of them and choose the right model and data in the right situation. Regarding the use of Wi-Fi data, the proposed framework of distinguishing pedestrians can be helpful in counting and tracking pedestrians' origin and destination on large scales. However, the framework lacks the implicit ability to track the exact trajectory followed by the user. The accuracy of the Wi-Fi sensors limits spotting pedestrian locations within a sphere of approximately 50 meters. The proposed model, although applicable to other sets of data with some modifications, still can be updated to more data-agnostic models in order to be transferable to other data sources. In addition, data collection was limited to certain dates, and the effect of weather conditions on the accuracy of the model was not investigated. 

Regarding the virtual reality data used for the studies of this dissertation, several technological limitations exist, as VR is a rapidly developing technology and is in its early days. The effect of crowd and group behaviour on pedestrians, which has proven to be an important factor in determining an individual's behaviour, is still difficult to implement in the virtual reality environment. On the graphics and reality of the scenarios, significant improvements can be made to make the scenarios as naturalistic as possible. Even in a very realistic VR environment, the necessity of using special equipment (wearing headsets, using game-pads, etc.) reminds the participants that they are in an artificial scenario and can lead to biased behaviour. When using VR, it should always be noted that it is impossible to
replicate behaviours that participants would show under actual fear, stress or risk of physical injury. However, by adding to the level of realism in the experiments and collecting additional information, VR has a promising potential application as an alternative or complementary tool to Stated Preferences experiments.

Although the data collection campaign was designed to be as diverse as possible, VR technology is yet to be fully developed to be accessible to people with special needs. Moreover, some people may experience symptoms of nausea, fatigue, motion sickness, etc. in the immersive environment. Different levels of previous exposure to virtual reality or video games might affect how participants perform in VR experiments. We tried to address this limitation by the addition of training sessions before the experiment. However, different individuals might require different amounts of time to be accustomed to an immersive 3D environment. 

Model interpretation of machine learning frameworks is a relatively new topic of interest, and all the proposed methods in the literature have their drawbacks in different aspects. The interpretability methods used in this dissertation are as well prone to the fundamental problems of similar methods, and with the advances of these methods, a reconfiguration of the models can be conducted. 

While this research has tried to address pedestrian dynamics in smart cities in the context of novel datasets and methodologies, it has only been able to apply the methods to specific problems. The datasets and the methodologies have yet to be expanded substantially to include various other data sources and issues of pedestrian research. 

\section{Future work}
\label{chap7:sec3}
The research presented in this dissertation can be extended and built upon in important directions. For the passive collection of Wi-Fi data, incorporating other data sources, labelled and unlabelled, and providing data-agnostic hybrid models to detect and tract network users can be a direction to follow and achieve higher accuracies. By developing a mobile application and asking permission from the user for their data, the quality of the data can be increased. To enhance the response rates of users, incentives such as promoting greener modes of transportation, counting carbon footprints and providing points for monetary awards in exchange for more steps can be provided. Moreover, aside from detection and counting, the behaviour of pedestrians can be modelled to analyze the contributing factors in mode choice, pedestrian route choice, activities, etc., and to predict the next locations or activities of pedestrians.

In the context of smart cities, the interaction of automated vehicles and pedestrians is another part of this dissertation with various possible extensions. The virtual reality experiments conducted are aimed at providing a framework for a comprehensive investigation of the interactions between the two agents. Wait time and trajectory, under specific conditions, were the issues covered in this dissertation. Training an automated vehicle based on the collected data can be another direction to follow and develop pedestrian-friendly braking systems for AVs. Extending the VR experiments by including group crossings, accessibility services for people with disabilities, and better qualities in the graphics of the scenarios, is another important area that needs to be explored. Analyzing communications of vehicles and pedestrians and developing and testing new audio or visual communication can help the pedestrians crossing more conveniently. The effectiveness of such methods can also be tested in the virtual reality environment. 

Future research can further enhance the interpretability of the models by including interactive effects of variables to the outcome and developing other methods of machine learning interpretability. Machine learning models can only fully replace traditional models when they are able to provide the same level of interpretability. Despite the recent advances on the subject, interpretability methods are a thriving field, and new methods should be tested continuously. By providing solid interpretability results from different areas of the interactions between pedestrians and AVs, a more clear understanding of future urban areas and the dynamics of pedestrians in that context can be elaborated.

The final area for future work would be to adapt these machine learning algorithms and transfer the models developed in this dissertation to real-world in-vehicle sensor data made available by various AV manufacturers. As discussed in this dissertation, these datasets are yet relatively small and without all the required information to focus on the pedestrians. A benchmark dataset, including the data related to pedestrians, seems to be of vital importance to help researchers train and test their models in a more consistent and comprehensive way. Virtual reality data collected in this study can form an important part of this benchmark dataset. 


