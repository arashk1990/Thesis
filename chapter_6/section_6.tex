\section{Conclusions and Future Works}
\label{S:T6}
In this study, we explore the use of naturalistic virtual reality data and advanced machine learning model to address pedestrians' crossing trajectory prediction. In our proposed method, contextual information from the environment are used as auxiliary data to enhance the accuracy of the model. By adding auxiliary data, our framework will be able to take into account the effects of road specifications, i.e. lane width and type of road, traffic parameters, i.e. speed limit, arrival rate, and environmental conditions, i.e. weather conditions and time of the day. All the auxiliary variable are chosen in a way that a hypothetical autonomous vehicle can observe and use the information on its prediction algorithm.

Pedestrian trajectory prediction can be used in various autonomous contexts, e.g. self driving cars or autonomous delivery robots. By having higher level estimations on the future behaviour of pedestrians based on their current behaviour, we can ensure a safe and comfortable trip for both pedestrians and passengers in the vehicles.

Our study, at its current stage, is not without limitations, most of which are currently being addressed and will be presented in the ICMC conference presentation. Providing a better understanding of the auxiliary variables and their effect of pedestrian behaviour, as well as a comparison of the model to current state of the art methods in the literature is under preparation at the time of the submission of this paper. In the future steps of the study, the two-way communication and training of the autonomous vehicle can be investigated by using game theory based approaches. Finally, a comprehensive framework consisting of pedestrian intention decisions and behaviours can be covered in future phase of the study, involving results from our wait time analysis research.