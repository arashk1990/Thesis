\thispagestyle{plain}
% \section*{Abstract}

\begin{table}[!h]
\centering

\begin{tabu}{X[0.1,c] X[0.5,c] X[0.1,c]}
    \addcontentsline{toc}{chapter}{Abstract}
    &\textbf{Abstract}&\\[1em]
    &Pedestrian Dynamics: Ubiquitous Sensing, Interactions, and Models&\\[0.7em]
    &by&\\[0.7em]
    &Arash Kalatian&\\[0.7em]
    &PhD in Civil Engineering - Department of Civil Engineering&\\[0.7em]
    &Ryerson University&\\[0.7em]
    &2020&\\[1em]
    &Dr. Bilal Farooq, Supervisor&\\[1em]
\end{tabu}
\end{table}

\begin{abstract}
The dissertation presents novel data sources and data-driven modelling frameworks for understanding the emerging pedestrian dynamics in response to the technological changes ushered by smart cities. It explores the influence of emerging technologies on the dynamics of urban areas from a pedestrian-oriented perspective.
To evaluate this influence, novel data sources and modern data-driven approaches replace traditional data collection methods and modelling techniques.
Part of the dissertation establishes new tools and techniques to predict and explain pedestrian behaviour that concerns the future dynamics in urban areas.
Specifically, we consider: (i) what are the novel data sources that can capture detailed information on pedestrian behaviour and how can we use them efficiently?
(ii) how to extract valuable information from these new high-dimensional data sources? 
(iii) how to utilize the data and tools to predict pedestrian behaviour in the future urban environment? and
(iv) how to interpret and explain pedestrian behaviour in the context of smart cities?

This thesis is based on four articles introduced in \cref{chap2,chap3,chap4,chap6}. \cref{chap2} introduces a semi-supervised residual network for transportation mode detection using passively collected labelled and unlabelled WiFi signal data.
\cref{chap3} provides a survival analysis to model the wait time behaviour of a crossing pedestrian and analyze the effect of smartphone distraction on pedestrian behaviour. Virtual reality is used in this chapter as a means of data collection in a controlled environment.
\cref{chap4} highlights pedestrian crossing behaviour in the presence of automated vehicles. A large virtual reality data collection campaign is designed and conducted to understand pedestrian behaviour in futuristic scenarios. Data-driven survival analysis is developed and applied to the data from virtual reality experiments to analyze pedestrian wait time before mid-block unsignalized crossings. By using a post-hoc model interpretation, the contributing factors to pedestrian wait time are assessed.
In \cref{chap6}, a neural network architecture is developed to incorporate sequential time-series data and contextual information to predict pedestrian trajectory. The proposed framework is applied to the virtual reality dataset. By interpreting the results, the environmental information that adds to the uncertainty to pedestrian trajectory is investigated.

\end{abstract}
\clearpage
