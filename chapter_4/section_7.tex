\section{Conclusion and future direction}
\label{S:Con}
In this paper, we investigated factors affecting pedestrians' wait time before crossing unsignalized crosswalks. With the upcoming revolutionary technologies on the roads, it is of vital importance to re-think and re-asses pedestrian behaviour in the presence of these unprecedented technologies. We tried to fill the existing research gap of investigating factors that appear to be significant contributions to pedestrians' crossing behaviour. Studying new technologies require new tools, thus, a virtual reality-based immersive and dynamic experiment was introduced in this study to obtain high-dimensional data from customizable scenarios in a safe way. Over a period of five months, a total of 180 people, selected in a heterogeneous and inclusive way, participated in our experiments. Participants were asked to cross an unsignalized crosswalk in various scenarios, incorporating speed limit, lane width, vehicles' arrival rate, road type, level of automation, braking system of cars, minimum gap between vehicles, weather conditions, and time of the day. Moreover, demographic information and travel habits of participants were collected using questionnaires before each experiment.

To analyze the effects of different covariates on pedestrians' wait time, survival analysis models were developed in this study. A traditional Cox Proportional Hazards (CPH) model as the baseline model, and a neural network based CPH model empowered by feature selection and interpretability methods were developed and trained in this study. Using an interpretable framework helps us capture nonlinearities among high-dimensional data, resulting in better goodness of fit achieved compared to the baseline model. Moreover, our framework outperforms simple neural-network based CPH models by following an embedded algorithm for systematic feature selection. Using a game theoretic-based interpretability method, we seek to replace traditional CPH methods by fulfilling their power in capturing the effects of covariates on baseline hazards function.

Based on the interpretability results of our study, we suggested some key practical implications and policy recommendations. Our results showed that pedestrians wait longer before crossing in the presence of automated vehicles. Wider lane widths, higher traffic densities, poor sight distances, and lack of walking habits in pedestrians found to be the other contributing factors to pedestrian wait time. Widespread educational campaigns before introduction of AVs on streets, enhanced safety measures on AVs, promoting active transportation modes, incorporating pedestrian-friendly infrastructures on streets and using pedestrian-to-vehicle communications are some of the possible solutions for future urban areas.
In conclusion, this study tries to contribute to the current literature in three general aspects:
\begin{itemize}
    \item Utilizing immersive virtual reality tools for relatively large-scale data collection, backed by a systematic design of experiment to optimize the information that can be inferred from the data. To the best of our knowledge, our VR data collection campaign is one of the biggest such campaigns for pedestrians studies. 
    \item Developing a neural-network-based survival model, to analyze the effects of different parameters on pedestrians' wait time. Incorporating neural network within a CPH model, we achieved an improvement in accuracy by 5\% compared to linear CPH models. 
    \item Using SHAP, as a modern game theoretic-based approach for neural network interpretation. Interpreting neural networks is gaining more popularity in recent years, in attempts to transfer black-box models to explainable models that can be used for policy and decision making. Limited number of studies have touched model interpretability in transportation.  
\end{itemize}

Our study was not without limitations. In terms of the data collection, as stated in the paper, participants training procedure can be enhanced to decrease the effect of previous VR experience on the performance. Providing accessibility services for people with disabilities to have more inclusive data collection is another important direction that needs to be followed. Enhancing quality of scenarios, including a broader investigation of lighting/weather condition, and different types of crossing, e.g. signalized and unmarked crossings, can be important directions to follow. Moreover, the effect of AV can be better validated by having an actual human driver for HDV scenarios, and conducting post-experiment surveys from participants to validate if they could distinguish between the two types of vehicles. Other methods of design of experiment may be also tried to compare the performance of the proposed D-Optimal design introduced in this study. Regarding the model, other deep networks can be utilized within or independent of the CPH model. Predicting pedestrian wait time using developed hazard functions can be addressed in future studies, which requires developing a time-dependant baseline hazard function. Current findings can be useful in the development of new audio, visual or direct vehicle-to-pedestrian communication methods that can make the pedestrian crossing more convenient. The effectiveness of such methods can also be tested in the virtual reality environment. Our paper is a part of an on-going research on interactions of humans and automated vehicle, which aims to develop prediction and training tools to develop socially-aware automated vehicles. Other directions of research, such as investigating crash and incidents, safety measures, etc. can be studied in future research using more comprehensive data available on the topics, or developing crash related scenarios in virtual reality.