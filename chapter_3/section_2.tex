\section{Background}
\label{S:Dback}
Despite the growing interest in pedestrian behaviour in the literature, pedestrian waiting time at intersections especially for unsignalized and mid-block crosswalks has yet to be widely discussed. For unsignalized intersections, Hamed~\cite{hamed2001analysis} developed a cox proportional hazards model to identify factors affecting waiting time and number of unsuccessful attempts required before a safe crossing. Their Results suggest that having accident experience, car ownership, number of people on the crosswalk, age, gender, type of trip and vehicle gap time are important factors in determining wait time before crossing. However, waiting time at signalized intersections has drawn more attention in the literature, mainly by looking into calculation of signal timings based on pedestrians’ waiting times. In 2003, Keegan and O’Mahony~\cite{keegan2003modifying} studied the effects of different countdown timers on pedestrian waiting time at signalized intersections. In 2013, Li~\cite{li2013model} developed a model for intended waiting time at signalized intersections taking into account bounded waiting times. Results showed that a large proportion of pedestrians cross the street immediately after they arrive at the intersection. In terms of the effects of waiting time on pedestrian violations, Brosseau \textit{et al.}~\cite{brosseau2013impact} analysed video data of signalized intersections and identified several factors that contribute to dangerous crossing, including maximum waiting time and clearance time.  

 Data in the aforementioned studies were collected either by questionnaires or by observing pedestrians for a short period of time. In addition, most studies on the subject consider waiting time at signalized intersections, mainly to explore the thresholds of pedestrian crossing signal timings. However, as it may be unsafe to track the pedestrian behavior while they are distracted using their phone, the distraction caused by smartphones has not been studied widely in previous literature. 
 
 Recent developments in virtual reality technology have made it possible to analyze different behaviours of pedestrians with minimal risks. Studies suggest that spatial knowledge developed in virtual environments resemble that of the actual physical environments~\cite{o1992effects,ruddle1997navigating}. Virtual Reality has been used in transportation studies in fields such as route choice or evacuation behaviour. However, these studies often lack the interaction element that can be implemented in Virtual Immersive Reality Environments (VIRE). Using VIRE, participants are immersed in an environment where vehicles respond to their actions. For instance, when a pedestrian walks in front of a car, the cars start to slow down, and if necessary, stop for the pedestrian to cross. Although experiments conducted in laboratory environments may lack the realism that is necessary for data collection, VIRE provides researchers with a safe environment that is close to real life situations. Considering the advantages of utilizing virtual reality environment for pedestrian-related experiments, data collection procedure was designed based on a virtual reality-based tool developed by LiTrans and introduced in~\cite{farooq2018virtual}. 
