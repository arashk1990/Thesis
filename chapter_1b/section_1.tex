\section{Data collection methods in pedestrian studies}
\label{chap1b:sec1}
\subsection{Location aware technologies}
Conventionally, customized surveys have been conducted to gather required sample for pedestrian datasets. Intrinsic problems of traditional methods include involvement of human errors and biased responses, as well as being time-consuming, expensive and not representative. As a result, using location-aware technologies to collect data have gained popularity among scholars in different fields~\cite{farooq2015ubiquitous}. These modern approaches in data collection have led to a dramatically growing interest in utilizing these data for mobility purposes. Location aware technologies that have been used for the purpose of detecting, tracking, or counting pedestrians in the relevant research studies include: body worn sensor, GPS data, GSM data, Wi-Fi data, Bluetooth transceivers data and RFID tags data. In general, these data collection approaches can be divided into two main categories: 1. User-Centric approaches and 2. Network-Centric approaches. User-Centric approaches, such as GPS or body worn sensors, require users to be actively involved in data collection procedure. Zheng \textit{et al.} used GPS data solely to detect mode of transportation~\cite{zheng2008understanding}. Defining features such as heading change rate, stop rate and velocity change rate and considering the conditional probability between different modes of transport, a graph-based post processing algorithm was proposed to detect network users' mode. To add to the accuracy of the results, some researchers have used multiple data sources simultaneously. Reddy \textit{et al.}, for instance, implemented GPS data along with smartphones accelerometer data to distinguish users movements between walking, running, biking and motorized traveling~\cite{reddy2008determining}. In another study, Stenneth \textit{et al.}, added data from transportation network to determine users' mode of transport between stationery, walking, biking, driving, and using public transit~\cite{stenneth2011transportation}. Although GPS methods provide high accuracy detection, requiring users intervention, high battery consumption and difficulties for wide-spread use of such methods, have led scholars to use other, although less accurate, sources. Turning on device's GPS and installing and running a mobile application to collect GPS records, make it hard to use these data in real life mobility problems.

 In Network Centric Approaches, such as GSM or Wi-Fi on the other hand, data can be collected passively with no intervention from users. Using cellular networks data resolves the problem of battery consumption and users intervention. The basic idea behind users' geolocalization with GSM data is to acquire their location based on the learning of which Based Transceiver Stations (BTS) they are connected to in specific time spans. Sohn \textit{et al.}, for example, used coarse grained GSM data to determine whether a user is staying in a place, walking or driving~\cite{sohn2006mobility}.  Wang \textit{et al.} used coarse grained call detail records to infer transportation mode between pairs of defined origins and destinations~\cite{wang2010transportation}. Travelers were clustered into three groups using K-Means algorithm: walking, public transit and driving. Low positioning accuracy, ping-pong handover effect and privacy concerns have been mentioned in this study as some of the main problems of using GSM data. In addition, relatively low density of BTS in some areas make GSM data an unreliable source for detecting mode specially at local levels. For trips within a block or neighborhood, for instance, GSM data cannot be used as a BTS cell size is at least 200 meters \cite{kalatian2016travel}.
 
 By implementing sensors with the ability to collect connection information from  Wi-Fi enabled devices, location and movement data of users can be inferred passively with no intervention from users~\cite{farooq2015ubiquitous}. Wi-Fi enabled devices can be discovered when they are inside Wireless Local Area Network. Being able to detect devices within access point's wireless range with no need of logging on, makes passive collection of data feasible. Mun \textit{et al.} coupled Wi-Fi and GSM data to train a decision tree classification algorithm for detecting various modes of transportation~\cite{mun2008parsimonious}. Features used for classification in this experiment include: Wi-Fi signal strength Variance, duration of dominant Wi-Fi access point, number of cell IDs that device connects to and residence time in cell footprint. Using Wi-Fi data, tracking indoor movements is also feasible. Various research studies have been conducted for inferring motion in indoor environments. Krumm \textit{et al.}, for instance, used Wi-Fi signal strengths and their variance as inputs to a Hidden Markov Model for smoothing transitions between the inferred states of still and moving~\cite{krumm2004locadio}.
 
 \subsection{Controlled experiments for data collection}
 When observing and tracking the behaviours of individuals are more important than the aggregate behaviours of the users, controlled experiments need to be done to assess the behaviours of pedestrians under hypothetical conditions. To study perception of people, pictures, maps and videos were used in the past. Recent developments in Virtual Reality VR environments have made investigation of human perception and behaviour possible . Users are immersed in VR environment in a controlled condition, which allows evaluation of their perception and behaviour. Combination of the physical environment and virtual elements, i.e. information or images, broadens the opportunities for content delivery~\cite{farooq2018virtual,jennett2008measuring,animesh2011odyssey,nah2011enhancing,faiola2013correlating}. Experiments in VR environments have successfully been conducted in different fields in cognitive studies \cite{farooq2018virtual}. People can develop realistic spatial knowledge similar to that of actual physical environments in the VR environment~\cite{o1992effects,ruddle1997navigating}. Physical reactions of users can also be recorded in VR environments by using electrocardiography, skin conductance or electroencephalogy and eye tracking. In the field of transportation, researchers have used VR in pedestrian route choice studies and their reaction to information in evacuation scenarios. However, these studies mainly lack the interactive and dynamic potential of VR. More specifically, user's actions in the VR environment needs to get responses from the elements of the environment, and vice versa. 

One of the main concerns that are raised for using virtual reality is the level of realism involved and how the results obtained through VR resemble those obtained from behaviours in real life. Several researchers have investigated such comparison. In studies on pedestrian behaviour, Bhagavathula \textit{et al.} compared different elements of pedestrian behaviour in virtual reality and real environments~\cite{bhagavathula2018reality}. Their comparison resulted that crossing intention, perception of safety, perception of risk and perception of distance were not significantly different in the two environments. On the other hand, it appeared that perception of speed in the two approaches of data collection were different among the participants. Similarly, Deb \textit{et al.} compared the behaviour of pedestrians crossing signalized intersection and concluded that both objective and subjective measures were similar in the participants in the two environments~\cite{deb2017efficacy}. However, an 11\% failure in completing the VR experiments was observed during their data collection, due to motion sickness. In another study focused on body movements of pedestrians crossing in virtual reality environments, Kalantarov \textit{et al.} concluded that wait time measures in their virtual realty experiment was in line with the laboratory studies and field observations~\cite{kalantarov2018pedestrians}.




 


