\section{Introduction}
\label{S:w1}
In transportation studies, mode detection is of interest as it helps city planners and transportation agencies to observe and track shares of different transportation modes over time. This information can then be exploited for planning, designing, and operating multimodal infrastructures required by traffic network users. Information derived based on modes can also be utilized in other fields, such as contextual advertisements, health applications (e.g. steps and calorie counters) and environmental studies (e.g. carbon footprints).

To infer transportation mode, self-reported surveys have conventionally been the main source for collecting transportation data from network users. Although these methods have been employed for decades, their intrinsic problems, as well as the recent advances in location-aware technologies, have made researchers rethink conventional travel survey techniques~\cite{chen2010evaluating}. Some of the main problems of traditional surveys include: being time-consuming, expensive and not representative, and involvement of human error and biased responses~\cite{murakami2004using,gong2012gps,stopher2007household}. Location-aware technologies and networks, on the other hand, can potentially be used for ubiquitous data collection at large scales and in different conditions. High penetration rate of smartphones allows collecting data from an ample number of users, who may not even be aware of the experiment. For instance, by using Wi-Fi sensors, data can be collected from all the participants carrying a Wi-Fi enabled device in the area of the experiment~\cite{farooq2015ubiquitous}. As Wi-Fi and Bluetooth sensors are already operational in some urban areas, e.g. city of Toronto~\cite{opendata}, no additional infrastructural costs are required for such studies.


Supervised learning algorithms have been the dominant tools to infer mode of transportation from the collected data in the literature. In such case, labelled records for each trip, i.e. mode of transportation for each trip, are required to train and validate algorithms. Thus, data should either be manually labelled by looking into video footage, for instance, or be limited to selected participants who would label their own trips. In both cases, the advantage of large scale data collection will be negatively affected as the number of data points obtained will be significantly reduced. To benefit from large amounts of unlabelled data available, semi-supervised or unsupervised algorithms can be taken into account. In basic terms, semi-supervised learning algorithms couple a small-sized labelled data with unlabelled data for the purpose of training a classifier~\cite{blum1998combining}. 

In this study, we develop a semi-supervised deep residual neural network for Wi-Fi signals to infer mode of transportation of network users in a congested urban area in Downtown Toronto. In recent years, Deep Neural Networks (DNN) have demonstrated successful performances in different fields of machine learning. DNNs, if coupled with adequate procedures, have shown impressive performances on complex and noisy data \cite{reed2014training,rolnick2017deep}. However, increasing the number of hidden layers after some point results in the problem of degradation in accuracy \cite{he2016deep}. This concern led to the introduction of impressively successful Deep Residual Networks (ResNet), which tries to increase the number of hidden layers by implementing \textit{Shortcut Connections} \cite{he2016deep}. 

The proposed framework, from data collection to model development and implementation, is a novel application of artificial intelligence techniques that has the potential to enable smart transportation planning, operations, and design using ubiquitous communication networks. The framework has been developed using Wi-Fi networks, but can also be ported to data coming from other future networks, e.g. 5G. In short, this study contributes to the current literature in three general aspects:
\begin{itemize}
    \item Collection and automatic labelling of ubiquitous Wi-Fi network data on congested roads of downtown Toronto, as the only source for transportation mode detection. This is potentially an attractive alternative to widely used GPS data that is dependent upon user's participation, e.g. installation of an App on their smartphone.
    \item Additionally, utilizing low-cost unlabelled Wi-Fi data by developing the framework in a semi-supervised manner. Passive data collection enables us to collect additional information from Wi-Fi network users, even if they are not directly part of the experiment.
    \item Development of very deep Residual Networks (100-150 layers), identifying the complexities and higher-order correlations within the labelled and unlabelled data, which results in higher prediction accuracy.
\end{itemize}
\noindent To the best of our knowledge, previous research in the area have not addressed semi-supervised ResNet for mode detection, or any other transportation studies incorporating ubiquitous data.

The rest of this paper is organized as follows: In the next section, we review previous researches on the subject. \cref{S:w4} presents data collection and pre-processing procedures in detail. We then describe the framework and architecture of our proposed algorithm in \cref{S:w3}. We apply our proposed framework on the data collected in \cref{S:w5}. In the end, conclusions and future research plans are outlined in \cref{S:w6}.
