\section{Conclusions and Future Works}
\label{S:t6}
Pedestrian trajectory prediction models can be used in various automated contexts, e.g., automated vehicles or automated delivery robots. By having better estimations of the future behaviour of pedestrians based on their current behaviour, we can ensure a safe and comfortable trip for both pedestrians and passengers in the vehicles as well as smooth traffic flow on urban roads.
In this study, we explored the use of naturalistic virtual reality data and advanced machine learning model to predict pedestrians' crossing trajectory. In the proposed method, contextual information from the environment are used as auxiliary data, and are added to sequential data of pedestrians' past trajectory, head orientations and distance to the upcoming vehicles, to train a LSTM network for predicting pedestrians' next coordinates. By adding auxiliary data, our framework takes into account the effects of road specifications, i.e. lane width and type of road, traffic parameters, i.e. speed limit, arrival rate, and environmental conditions, i.e. weather conditions and time of the day. All the auxiliary variable are chosen in a way that a hypothetical AV can observe and use the information on its prediction algorithm. The results showed that the incorporation of contextual information within the trajectory prediction models not only increases the accuracy in prediction, but also is beneficial in avoiding overfitting of the developed models to the training data. By implementing a neural network interpretability method, we conclude that a pedestrian-oriented AV dataset requires to be able to include diverse weather and vision conditions, as well as different traffic conditions in order to be able to predict and model the behaviour of pedestrians accurately. Despite the growing accessibility of open-access AV datasets, a major part of the currently available datasets fail to provide such variety in environmental conditions. AV manufacturers can also use our methodological framework and results to better understand the environmental factors that can negatively affect their prediction algorithms, and try to address the possible shortcomings by changing the focal point of their data collection efforts to include problematic situations.   

Our study, at its current stage, is not without limitations. A comparison of the model to current state of the art methods in the literature is a possible direction to follow. Benchmark datasets can be used to retrain our proposed model and apply similar interpretability methods to understand the contributing factors in different settings. Although the model was developed on virtual reality data, transferability of the methods to real AV datasets can be investigated in future studies. In the future steps of the study, the two-way communication and training of the AV can also be explored. Redefining the problem to include the vehicle side of the interaction with pedestrians and consider the comfort and safety of passengers is another possible dimension to discover in future studies.  Finally, a comprehensive framework consisting of pedestrian intention decisions and behaviours can be covered in future phase of the study.