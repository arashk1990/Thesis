\section{Conclusion}
\label{S:w6}
In this paper, a novel framework to infer transportation mode based on Wi-Fi data is introduced. By implementing \emph{URBANFlux} technology introduced in~\cite{farooq2015ubiquitous}, Wi-Fi communication records in a congested urban area in downtown Toronto were collected on two separate days, as labelled data from participants and unlabelled data from non-participants. After data preparation processes and extracting fifteen features based on time and speed, signal strength and number of connections, a ResNet Multilayer Perceptron Network was developed and implemented as the core of a Pseudo-label semi-supervised learning algorithm. The proposed framework enables us to exploit the ample amount of low-cost unlabelled data, which makes our framework's performance adaptable to real life scenarios. In addition, by implementing a ResNet-based architecture, we could successfully train networks with a large number of hidden layer, which helps us benefit from high-level features in our classification task. Moreover, ResNet architecture helped us utilize noisy Wi-Fi data, with highly correlated features, which can be extracted with low costs. By adding semi-supervised learning to our framework, we tried to tackle the problems and costs of data collection. Calibrating this framework, we reached a total accuracy of 82.9\% for mode detection. 

In this study, solely Wi-Fi data are utilized, and our approach does not rely on other data sources to infer transportation mode. Our approach, thus, is more cost-effective and easier to implement for real world applications, with no users' interventions, or processes like multi-source data fusion, required. A key point of this study is its capability of reasonably large scale implementation of Wi-Fi-based mode detection on actual urban roads, where real traffic exists. Bluetooth and Wi-Fi signal sensors are already implemented in a number of cities, including Toronto. Our results suggest that by defining cooperative projects among city policy makers and telecommunication sectors, intelligent systems can be further optimized to benefit both sides. One major implication of large scale implementation of our study in transportation planning would be the inference of operational traffic parameters (e.g. flow, density, delay), to find optimal routing suggestions in shorter times, dynamic traffic management, and smarter network designs. It is in particular significant as current methods do not provide detailed information on large scale state of different multimodal movements. labelling data collected from anonymous network users can be extremely difficult and expensive, if not impossible. By implementing our framework, these data can be used in their unlabelled form to re-train our classifiers, and subsequently, improve their accuracy. Another advantage of our framework is its capability of being implemented in urban areas with Bluetooth or Wi-Fi sensors already incorporated, with virtually no additional cost for sensors and infrastructures.

With regard to the high penetration rate of smartphones in recent years, \emph{URBANFlux system}, empowered by this method, can help city decision makers better observe and predict users' travel behaviours and their changes in short or long terms.  Transportation mode detection can also be useful in urban ubiquitous sensing, as it gives insight into energy consumption, pollution tracking and prediction and burned calorie estimation.

Our study is not without limitations, which can be explored as future research directions. One future direction can be the modification and application of the framework for other types of data sources to compare the results and develop a more data-agnostic and generic framework. Rapid development of smartphones and their respective infrastructure (e.g. 5G technology, next-gen GPS, Wi-Fi 6), require constant upgrades of the models. Modifications to the models can be smoothed by training models on various, or even hybrid data sources. Current study relied on two one-day data collection campaigns, limited to certain time of the day and traffic conditions. Continuous and more widespread data collections, taking into account larger areas with different demographics and traffic characteristics will be another possible direction to the current state of the study. Regarding real-world application of our framework in transportation management, lacking a separate class for public transit vehicles (e.g. bus, subway, street car) is a limiting aspect of our study. An extended and more inclusive data collection that uses Wi-Fi sensors installed in public transit stations, or incorporating either real-time or offline data sources available to major public transit providers, can be another extension to strengthen our framework's application. Moreover, underlying signal timing or network structure data, along with historical data from side-walks, arterial, and bike lanes can be added to further improve our knowledge on network users' modes of transportation. Real-time inference of transportation mode can also be investigated in future studies, adding unlabelled data from network users to update the framework continuously. Aside from the data, more advanced methods can also be investigated in the future studies. In particular, convolutional neural networks or generative adversarial networks can be incorporated to replace fully connected layers used in our framework. A common shortcoming of transportation mode studies is the incapability of implementing their profound networks in real-world applications, due to the high computational power required to develop such networks. Although we tried to address this issue by local training of the model, more accurate real-time results can be achieved by training model on streams of data collected continuously. An extensive analysis of the model efficiency can be a direction to follow in the future studies. \\ 